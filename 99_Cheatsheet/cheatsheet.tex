% !Mode:: "TeX:UTF-8"
\documentclass[UTF8,a4paper,landscape,compress]{paper}
\usepackage{ctex}
\usepackage{amsmath}
\usepackage{pdfpages}
\usepackage{graphicx}
\usepackage{wrapfig}
\usepackage{listings}
\usepackage{xcolor}
\usepackage{amssymb}
\usepackage{bm}
\usepackage{enumitem}
\usepackage{multicol}
\usepackage{float}
\usepackage{cuted}
\usepackage{titlesec}
\usepackage{fancyhdr}
\usepackage[landscape]{geometry}
\usepackage[figuresright]{rotating}
\linespread{0.5}
\geometry{left =0cm,right = 0cm,top = 0.5cm, bottom = 0.5cm}
\pagestyle{fancy}
\renewcommand{\subsection}[1]{{\small\textbf{\underline{#1}}}\\ }
\renewcommand{\section}[1]{{\normalsize\textbf{\emph{#1}}}\\ }
\newcommand{\List}[1]{\begin{itemize}[fullwidth,itemindent=0em] #1 \end{itemize}}
\begin{document}\footnotesize
\begin{multicols}{4}
\section{贝叶斯决策}
\subsection{决策理论}
\List{
    \item {贝叶斯公式: $\displaystyle{P(\omega_i|\vec x) = \frac{p(\vec x|\omega_i)P(\omega_i)}
    {\sum p(\vec x|\omega_j)P(\omega_j)}}$}
    \item {联合概率分布和条件概率分布:$$P(B|A)P(A) = P(B,A)$$}
    \item {最小错误率方法:$$\arg\max_i P(\omega_i|\vec x) = \arg\max_i P(\vec x | \omega_i)P(\omega_i)$$}
    \item {最小错误率分析:$P(e) = P(\omega_2)P_2(e) + P(\omega_1)P_1(e), P_2(e)$是把2分类错误成1的概率}
    \item {多类正确率分析: $P(c) = \sum P_i(c) P(\omega_j)$ 对应最高的曲线加先验的面积}
    \item {False postive: 将负样本判成正样本。False negative:将正样本判成负样本}
    \item {ROC曲线:横轴 False Postive,纵轴 True Positive}
    \item {最小风险贝叶斯决策:$\lambda(\alpha_i,\omega_j)$是在$\omega_j$状态上采用$\alpha_i$决策的代价。
    (选对了可以认为是0)。决策$\arg\min\sum_j\lambda(\alpha_i,\omega_j)p(\vec x | \omega_j)p(\omega_j)$}
    \item {限定一类错误率,求第二类错误率最小值:拉格朗日待定系数法 $\gamma = P_1(e) + \lambda(P_2(e) - \epsilon_0) = 1 - P_1(c) + \lambda P_2(e) - \lambda\epsilon_0$。对$\lambda$求偏导,一维情况得到决策面满足$\lambda = \frac{p(t|\omega_1)}{p(t|\omega_2)}$,决策面限定错误率达到指标上限。(图解:看残余面积)。}
    \item {最小最大决策:风险已知,先验概率未知。$R = \int_{\mathcal R_1}\lambda_{11}p(\omega_1)P(\vec x|\omega_1) + \lambda_{12}p(\omega_2)P(\vec x|\omega_2) + \int_{\mathcal R_2}\lambda_{21}p(\omega_1)P(\vec x|\omega_1) + \lambda_{22}p(\omega_2)P(\vec x|\omega_2) = a + p(\omega_1)b$风险不可确定,以$b=0$作为最坏情况的最优解(防止$P(\omega_2)$产生变化}
    \item {多类决策:可以进行逐一比较。或者算最大的$P(\omega_i|\vec x) > P(\omega_j|\vec x)\Rightarrow p(\vec x|\omega_i)p(\omega_i) > p(\vec x|\omega_j)p(\omega_j) \Rightarrow {p(\vec x|\omega_i)}{p(\vec x|\omega_j)} > {p(\omega_j)}{p(\omega_i)} \Rightarrow \ln p(\vec x|\omega_i) + \ln p(\omega_i) > \ln p(\vec x|\omega_j)\ln p(\omega_j)$(对数似然)}
    \item {决策面以及两类情况的决策面:$g(\vec x) \begin{cases} >0 \Rightarrow \in \omega_1 \\<0 \Rightarrow \in \omega_2\end{cases}g(\vec x) = 0$是决策面}
}
\subsection{朴素贝叶斯决策和正态分布决策}
\List{
    \item {朴素贝叶斯:认为所有变量独立:$P(\vec x|\omega) = \prod_i p(x_i|\omega)$}
    \item {正态分布基本性质:$P(\vec x) = \frac1{(2\pi)^{d/2}|\Sigma|^{1/2}}\exp(-\frac12(\vec x-\mu)^{\mathrm T}\Sigma^{-1}(\vec x-\mu)$}
    \item {性质:等密度点是超椭球面$(\vec x-\mu)^{\mathrm T}\Sigma^{-1}(\vec x-\mu) = const$,马氏距离,不相关性$E_{12} = E_1E_2$ == 独立性$P_{12} = P_1 P_2$(仅在正态分布下)}
    \item {边缘分布和条件分布,线性变化正态性。$\vec y = \mathbf A \vec x \rightarrow N(\mathbf A\mu,\mathbf A\Sigma\mathbf A^{\mathbf A^T})$}
    \item {$\Sigma_i = \sigma^2\mathbf I$(协方差相等均对角),分类面为圆形(先验不相等)或中位(先验相等)}
    \item {$\Sigma_i = \Sigma$(协方差相等),分类面为椭圆形(先验不相等)或直线(先验相等,相对中位线发生偏转)}
    \item {其他可能性:二次曲线。计算错误率:公式,近似计算上界,实验估计}
}
\section{概率密度函数的估计}
\subsection{参数和非参数估计}
\List{
    \item {参数空间,点估计,区间估计,统计量等概念。MLE方法$\max \sum_i \ln(p(\vec x_i|\theta))$(有的时候不需要求导,}
    \item {正态分布的MLE估计方法:$\mu = \frac1N\sum_i\vec x_i, \Sigma = \frac1N\sum(\vec x - \mu)(\vec x - \mu)^{\mathrm T}$}
    \item {可识别性:参数空间内每两个不同参数均有不同的概率密度函数,连续的往往是可以识别的,离散的往往不能识别}
    \item {贝叶斯参数估计:将参数认为成随机变量,并有先验。$P(\theta|\vec x) = \frac{P(\vec x | \theta)p(\theta)}{p(\vec x)}$}
    \item {正态分布MAP:$\vec x \sim N(\mu,\sigma^2), p(\mu) \sim N(\mu_0,\sigma_0^2)$(均值有不确定性)$P(\mu|\vec x) = \frac1{p(\vec x)}\prod \frac1{\sqrt{2\pi}\sigma}\exp(-\frac12\frac{\vec x_k - \mu}{\sigma}^2\frac1{\sqrt{2\pi}\sigma_0}\exp(-\frac12\frac{\vec mu - \mu_0}{\sigma_0}^2) \sim N(\mu_N,\sigma_N),\mu_N = \frac{N\sigma_0^2}{N\sigma_0^2+\sigma^2}m_N + \frac{\sigma^2}{N\sigma_0^2+\sigma^2}\mu_0$。
    
    结果仍然是正态分布。$N=0\rightarrow$仅能靠先验,没有实验信息。$N\rightarrow \infty$实验信息足够多,先验信息没有用。$\sigma_0\rightarrow 0$先验太强,实验信息被忽视。$\sigma_0\rightarrow\infty$先验信息太弱,被忽视。
    
    当先验可靠时,利用更多的信息。目的是$\max P(\vec x | \theta)p(\theta)$。同等情况下MLE更简单}

    \item {最小二乘等价于对认为误差是正态分布的MLE方法,此方法上的加参数先验为正态分布的MAP方法就是L-2正则化}
    \item {Parzen窗法:窗口大小的选择,正态分布Parzen窗}
    \item {三类误差:贝叶斯误差:特征一旦选定就无法改变的误差。模型误差:模型不标准或者错误。估计误差:参数估计的时候产生的误差。维数问题:普遍x100个样本一维。依靠PCA,独立性等方式降低维数灾难}
    \item {过拟合:MLE的普遍特性,考虑增加样本或者引入MAP方案,通过简单模型(参数化模型,对角矩阵,共享参数)降低参数}
    \item {错误率估计:先验未知$\epsilon = \frac kN, E(k) = N\epsilon, Var(k) = N\epsilon(1-\epsilon),Var(\hat\epsilon) = \frac{\epsilon(1-\epsilon)}{N}$。先验已知:$E(\hat\epsilon) = \epsilon, Var(\hat\epsilon) = \frac1N\sum P(\omega_i)\epsilon_i(1-\epsilon_i)$。已知先验方差更低。交叉验证和留一法等}
}
\subsection{EM, GMM 算法}
\List{
    \item {GMM算法:$P(X|\Theta) = \sum\alpha_ip_i(X|\theta_i),\sum \alpha_i = 1, p_i(X|\theta_i)\sim N(\mu_i,\sigma_i) \text{log-MLE}: \sum_i\ln\sum_jN(x_i|\mu_j,\Sigma_j)P(\omega_j)$}
    \item {已知概率是多类高斯混合而成的,每一类的比例(先验)未知,每一类高斯的参数未知}
    \item {$P(\omega_k|x_i,\mu_k,\Sigma_k) = \frac{N(x_i|\mu_k,\Sigma_k)P(\omega_k)}{\sum_j N(x_i|\mu_j,\Sigma_j)P(\omega_j)}$}
    \item {$\hat P(\omega_k|x_i,\mu_k,\Sigma_k) = \frac1N\sum_i P(\omega_k|x_i,\mu_k,\Sigma_k)$}
    \item {对$\mu_k$求偏导为零得到$\hat\mu_k = \frac{\sum_i P(\omega_k|x_i,\mu_k,\Sigma_k)x_i}{\sum_i P(\omega_k|x_i,\mu_k,\Sigma_k)}$}
    \item {对$\Sigma_k$:$\hat\Sigma_k = \frac{\sum_i P(\omega_k|x_i,\mu_k,\Sigma_k)(x_i-\mu_k)(x_i-\mu_k)^{\mathrm T}}{\sum_i P(\omega_k|x_i,\mu_k,\Sigma_k)}$}
    \item {解决策略:E步利用上一步的东西计算$P(\omega_k|x_i,\mu_k,\Sigma_k)$,M步计算参数迭代}
    \item {EM 算法:解决数据缺失或隐变量的问题$X$显变量,$Y$隐变量。$P(X,Y|\theta) = p(Y|X,\theta)p(X|\theta)$}
    \item {$L(\theta) = \ln p(X|\theta) = \ln (\sum_Yp(X,Y|\theta))$,设$Y\sim q(Y)$}
}
\section{线性判别函数}
\List{
    \item {线性判别,广义线性判别,增广向量等}
    \item {感知准则函数:假设样本线性可分,$\begin{cases}
        a^{\mathrm T}y > 0 \text{对一切}y_i \in \omega_1\\
        a^{\mathrm T}y < 0 \text{对一切}y_i \in \omega_2
    \end{cases}$
    采用$y_i' = \begin{cases}
        y_i\\
        -y_i
    \end{cases}$
    得到$\forall i, a^{\mathrm T}y_i' > 0$,错分样本集合设为$Y^k$}

    \item {目标函数$J_P(a)  = \sum_{y\in Y^k}-a^Ty \ge 0$,希望优化到0。梯度下降得到$a{k+1} = a_k + \rho\sum_{y\in Y^k}y$}
    \item {1-bit SGD, $\rho = 1$ 可以达到最小值(可以硬证明)}
}
\section{SVM}
\List{
    \item {优化问题:$\min \frac12 w^{\mathrm T}w, \text{  subject to  } d_i(w^{\mathrm T}x_i + b) > 1$,凸优化问题加拉格朗日乘子转对偶方法优化}
    \item {$J = \frac12 w^{\mathrm T}w - \sum \alpha_i[d_i(w^{\mathrm T}x_i + b) - 1],\alpha_i > 0$ 转优化$J(\vec W, \vec \alpha)$。}
    \item {鞍点是最值点:$J(\vec W',\alpha) \le J(\vec W',\alpha') \le J(\vec W,\alpha')$,左侧得到$\sum (\alpha_i' - \alpha)g_i(\vec w') \le 0 \Rightarrow \sum g_i(\vec w') >0$说明是可行点,取$\alpha_i = 0$ 得到$\alpha_i'g_i(\vec w') = 0$}
    \item {右侧:$f(\vec w') \le f(\vec w) - \alpha'g(\vec w) \le f(\vec w)$得到最优点(强对偶条件)}
    \item {对偶求鞍点:$w = \sum \alpha_id_ix_i,\sum alpha_id_i = 0$}
    \item {对偶方法:$\max Q(\alpha)= \sum \alpha_i - \frac12\sum\sum\alpha_i\alpha_jd_id_jH_{ij},H_{ij} = x_i^{\mathrm T}x_j$,之后可以更换为核函数。}
    \item {线性不可分松弛:$\min \frac12 w^{\mathrm T}w + C\sum\xi$。松弛$d_i(w^{\mathrm T}x_i + b) > 1 - \xi_i$,C越大,要求性能越好,间隔越小,C越小要求间隔越大。}
    \item {非线性核函数方法,多项式$K(x,y) = (xy+1)^p$,RBF:$\exp(-\|x-y\|^2 / 2\sigma^2)$等}、
    \item {采用L-1正则化和L-2正则化的函数的SVM,L-1稀疏学习,L-2防过拟合}
    \item {产生式模型(优点:可以提供大量信息量)和判别式模型(优点:简单)}
}
\section{神经网络}
\List{
    \item {损失函数:TSSE:$\frac12(d_i - y_i)^2$,可以进一步补充为对整个batch的求导}
    \item {BP算法,矩阵求导,反向传播,链式法则等}
    \item {CNN,RNN,LSTM的结构(有空补上)。序列长度:一个sample的长度(sample之间没有关系)}
}
\section{决策树}
\List{
    \item {交叉熵:$E = -\sum p_i\ln p_i$}
    \item {Gini 不纯度:$1 - \sum P(w_n)^2$}
    \item {误差不纯度:$1 - \max_j P(\omega_j)$}
    \item {信息熵的增量:$l - p(l)l_l - p(r)l_r$(左右两支加权计算)}
}
\section{近邻法和距离}
\List{
    \item {最近邻法错误率在1\~2倍贝叶斯错误率中间}
    \item {压缩近邻法:首先用近邻法测试所有测试样本,如果测错了放进评测集,通过这种方式扩充评测集}
    \item {距离度量:1. s-Minkowski: $D(x,y) = [\sum|x_i - y_i|^s]^{1/s}$。$s=2$为欧几里得距离。Chebyshev 距离:$\max_i |x_i - y_i|$,马氏距离$(x - y) ^ {\mathrm T}Q(x - y)$}
    \item {距离的正定性,三角性,Holder不等式等(不会)}
    \item {KL 散度作为概率PDF之间距离,切距离:做流形变换之后点到直线的距离(单边切距离,双边切距离)}
}
\section{特征提取和特征选择}
\subsection{PCA, KL变换等线性方法}
\List{
    \item {目的:防止维数灾难,方便可视化和理解}
    \item {Fisher 方法:找一个方向使得投影结束后类间方差最大,类间方差最小。类间方差$(m_1 - m_2)(m_1 - m_2)^{\mathrm T}$。类内方差$\sum_j \sum_i (x_{ij} - m_j)(x_{ij} - m_j)^{\mathrm T}$}
    \item {投影之后方差变化做比$J(w) = \frac{w^{\mathrm T} S_bw}{w^{\mathrm T} S_ww}$最大转拉格朗日乘子:$L = w^{\mathrm T} S_bw - \lambda(w^{\mathrm T} S_ww-c)$(控制一个类内方差不变)得到$w$是$S_w^{-1}S_b$的最大特征值对应的特征向量。}
    \item {Fisher投影之后的分类问题:按照两类重心的重点、或者按照加权重心或者$\frac{m_1 + m_2}2 + \frac{\ln P(w_1) - \ln P(w_2)}{N_1 + N_2 - 2}$先验。也可以采用Bayes决策手段}
    \item {其他的Fisher手段,局部方法,非线性方法等,选取其他判据方案等(如矩阵迹)}
    \item {多维度Fisher方法:选取最大的几个向量}
    \item {PCA:最小化剩余方差$\min E(x - \hat x)^{\mathrm T}(x - \hat x) = E[\sum_{i=d+1}^\infty c^2]$}
    \item {PCA:找$S$(对称正交矩阵)的从大到小的特征值,对应的特征向量组成变换矩阵。}
}
\subsection{特征选择方案}
\List{
    \item {向前向后方法,顺序前进法,顺序后退法(找收益最大的和损失最小的)}
    \item {Relief方案,遗传算法等,存在问题:特征选择过学习}
}
\subsection{MDS, LLE, ISOMap,遗传算法}
\List{
    \item {MDS方案:给定两点之间的距离,找$p$维的$x$使得$d_{rs}^2 = (x_r - x_s)^{\mathrm T}(x_r - x_s) = x_r^{\mathrm T}x_r + x_s^{\mathrm T}x_s -2 x_r^{\mathrm T}x_s$可以构造$B = XX^{\mathrm T}, n \times n$,进行$p$维PCA提出 $B = V\Lambda V^{\mathrm T}, X = V_1\Lambda^{\frac12}$,展开成$P$维空间}
    \item {ISOMAP 方案:先通过kNN或者$\epsilon$圆构建一个整个图的支撑树,之后采用Dijkstra算法找到图中任意两点的最近距离
    构建距离矩阵,之后使用MDS方法确定坐标}
    \item {LLE(局部线性化):先通过kNN或者$\epsilon$圆找到每一个点的近邻然后把每一个面元铺平}
    \item {GA:初始化种群,交叉和编码方式等}
}
\section{Ensemble}
\List{
    \item {Adaboost:采用一组弱学习器进行学习。方案:1. 初始化$w_1(i) = \frac1N$. 2. 归一化:$p_l(i) = \frac{w_l(i)}{\sum_i w_l(i)}$ 3. $\epsilon_l = \sum_{mis}p_l(i)$ 4. $a_l = \frac12\ln\frac{1-\epsilon_l}{\epsilon_l}$ 5. 调整:$w_{l+1}(i) = w_l(i)\exp(\pm a_l)$正:错分样本加权重。负:正确样本减权重。5. 训练结束之后按$a_l$对$L$个分类器加权投票。}
    \item {随机森林:随机抽取若干特征构造树,按照准确率进行投票,投票悖论等}
}
\section{聚类分析,非监督学习方法}
\List{
    \item {GMM聚类:对样本进行GMM算法,每一个成分分成一类}
    \item {kmeans方案:对分割进行调整规则:$\frac{N_j}{N_j + 1}\|y-m_j\|^2 < \frac{N_k}{N_k - 1}\|y-m_k\|^2$将$y$从$k$类移到$j$类,复杂度$\mathcal O(n)$}
    \item {分类个数的选取:选择肘点(二阶差分最大(一阶差分变化最大))}
    \item {核函数方法,选择核函数代替欧几里得距离}
    \item {多级聚类方案:将两个最近/最远/平均距离最近的两类每次聚类,直到只剩下一类。形成聚类树。$\mathcal O(n^3)$复杂度}
    \item {谱聚类:首先构造相似距离矩阵$D$(越远的距离越大,可以使用RBF核或者其他的$\epsilon$圆或者kNN方案。将行相加构造对角矩阵$D$。$L = D - W$,$L_{rw} = D^{-1}L$构造非归一化和归一化的拉普拉斯矩阵。计算两个矩阵的前$k$个向量组成n行k列的矩阵,对此矩阵进行聚类。复杂度$\mathcal O(n^2)$}
}
\clearpage
\section{各种情况下的EM算法}
\subsection{一般情况下的EM算法}
\List{
    \item {$X$是显变量,$Z$是隐变量(一般来说是离散的)。联合分布$p(X,Z|\theta)$确定。目标是最大化$p(X|\theta) = \sum_Zp(X,Z|\theta)$}
    \item {强行引入$Z$的分布$q(Z)$,立即得到$\ln(p(X|\theta)) = \mathcal L(q,\theta) + KL(q\|p)$}
    \item {$\mathcal L(q,\theta) = \sum_Zq(Z)\ln\frac{p(X,Z|\theta)}{q(Z)}$}
    \item {$KL(q\|p) = -\sum_Zq(Z)\ln\frac{p(Z|X,\theta)}{q(Z)}$}
    \item {以上根据$\ln P(X,Z|\theta) = \ln P(Z|X,\theta) + \ln P(X|\theta)$,$KL(q\|p)$是KL散度,当且仅当$q(Z) = p(Z|X,\theta)$(估计完全正确)时等于0否则大于零,注意$\sum_Zq(z)\ln p(X|\theta) = \ln(p(X|\theta))$($\ln p(X|\theta)$与$Z$无关}
    \item {$\mathcal L(q,\theta) \le \ln(p(X|\theta))$,得到优化下界}
    \item {E-step:使用$q$最大化$\mathcal L(q,\theta)$,由于$\ln(p(X|\theta))$不变,故只能KL散度为0,$q(Z) = p(Z|X,\theta)$时$\mathcal L$达到最大化。}
    \item {M-step:使用$\theta$优化$\mathcal L$,这将使得$p(Z|X,\theta)$导致非零的KL散度,给E-step留空间。}
    \item {当总共有$n$个样本时,$P(Z|X,\theta) = \frac{P(X,Z|\theta)}{\sum_ZP(X,Z|\theta)} = \frac{\prod p(x_n,z_n|\theta)}{\sum_Z\prod p(x_n,z_n|\theta)} = \prod p(z_n|x_n,\theta)$。其中用到了交换求和求积顺序,同时需要求条件概率。}
}
\subsection{基于上述解释的EM算法}
\List{
    \item {E-step:计算$p(Z|X,\theta)$}
    \item {M-step:计算$\theta = \arg\max_{\theta}Q(\theta,\theta^{old}) = \sum_Zp(Z|X,\theta^{old})\ln p(X,Z|\theta)$。这里忽略了上面的分母因为和$\theta$无关}
}
\subsection{EM算法用于有先验的贝叶斯参数估计,GEM算法}
\List{
    \item {$\ln P(\theta|X) = \ln P(X|\theta) + \ln P(\theta) - \ln P(X)$进行优化}
    \item {GEM 算法:对M-step进行梯度下降而不是一次到最优点}
}
\subsection{GMM算法的EM表示}
\List{
    \item {E-step: $\gamma(z_{nk}) = \frac{\pi_k\mathcal N(x_n|\mu_k,\Sigma_k)}{\sum_j \pi_j\mathcal N(x_n|\mu_j,\Sigma_j)}$}
    \item {M-step: $\begin{cases}\pi_k = \frac1{N}\sum\gamma(z_{nk}), N_k = \sum\gamma(z_{nk})\\ \mu_k = \frac1{N_k}\sum\gamma(z_{nk})x_n\\ \Sigma_k = \frac1{N_k}\sum\gamma(z_{nk})(x_n - \mu_k)(x_n - \mu_k)^{\mathrm T}\end{cases}$}
    \item {对数似然:$\sum\ln\sum\pi_k\mathcal N(x_n|\mu_k,\Sigma_k)$}】
    \item {隐变量是二值的$Z$,$Z_{nk}$表征第$n$个样本是否在$k$的聚类中。$P(X,Z|\theta) = \prod_n\prod_k\pi_k^{z_{nk}}\mathcal N(x_n|\theta)^{z_{nk}}$}
    \item {$p(Z|X,\theta) = const \prod_n\prod_k\pi_k^{z_{nk}}\mathcal N(x_n|\theta)^{z_{nk}}$具有$X$在第k个聚类中的概率,可以用条件公式贝叶斯等}
    \item {M-step 必然得到GMM算法的结果}
}
\subsection{使用琴生不等式推出的EM算法}
\List{
    \item {$\ln P(x|\theta) = \ln\sum_Z P(x,z|\theta) = \ln\sum_zq(z)\frac{p(x,z|\theta)}{q(z)} \ge \sum_zq(z)\ln p(x,z|\theta) - \sum_zq(z)\ln q(z) = F(q,\theta)$}
    \item {E-step 仍然是对$F(q,\theta)$进行优化,得到结论仍是$q(Z) = p(Z|X,\theta)$,原因是此时处理之后$F(q,\theta) = \mathcal L(\theta)$。我们优化的是$q$}
    \item {M-step 优化$Q(\theta,\theta^{old}) = \sum_zp(Z|X,\theta^{old})\ln p(x,z|\theta)$和前面的相同}
}
\subsection{GMM对应一个batch的样本}
\List{
    \item {$P(Z|X,\theta) = \prod p(z_n|x_n,\theta)$,混合模型下$p(x,z|\theta) = \sum alpha_{z_i}p_{z_i}(x|\theta)$。注意这里的表示方式和上面的二值表示不同,这里表示的是下标参数,因此$\sum_Z$需要展开成那个很复杂的式子。设$y_i = k$表示的是$i$样本在$k$类中,直接得到M步为
    
    $\sum_{y_1 = 1}^M \cdots\sum_{y_1 = 1}^M\sum_{i=1}^Nln(\alpha_{z_i}p_{z_i}(x|\theta_{z_i}))$}。
    \item {这样做没有上面那么做好,二值的可以表示的更简洁。求解过程中注意区分$\theta,\theta^{old},p(z|x,\theta_old)$中不含我们需要的东西}
}
\clearpage
\end{multicols}
\end{document}