\documentclass{article}
\usepackage{mathrsfs}
\usepackage{amsmath}
\usepackage{amsthm}
\usepackage{amssymb}
\usepackage{graphicx}
\usepackage{color}
\usepackage{bm}
%\include{macros}
%\usepackage{floatflt}
%\usepackage{graphics}
%\usepackage{epsfig}


\theoremstyle{definition}
\newtheorem{theorem}{Theorem}[section]
\newtheorem{lemma}[theorem]{Lemma}
\newtheorem{proposition}[theorem]{Proposition}
\newtheorem{corollary}[theorem]{Corollary}

\theoremstyle{definition}
\newtheorem*{defition}{Definition}
\newtheorem*{example}{Example}

\theoremstyle{remark}
\newtheorem*{remark}{Remark}
\newtheorem*{note}{Note}
\newtheorem*{exercise}{Exercise}

\setlength{\oddsidemargin}{-0.25 in}
\setlength{\evensidemargin}{-0.25 in} \setlength{\topmargin}{-0.25
in} \setlength{\textwidth}{7 in} \setlength{\textheight}{8.5 in}
\setlength{\headsep}{0.25 in} \setlength{\parindent}{0 in}
\setlength{\parskip}{0.1 in}

\newcommand{\homework}[4]{
\pagestyle{myheadings} \thispagestyle{plain}
\newpage
\setcounter{page}{1} \setcounter{section}{#4} \noindent
\begin{center}
\framebox{ \vbox{\vspace{2mm} \hbox to 6.28in { {\bf
THU-70250043,~Pattern~Recognition~(Spring 2017) \hfill Homework: 1} }
\vspace{6mm} \hbox to 6.28in { {\Large \hfill #1 \hfill} }
\vspace{6mm} \hbox to 6.28in { {\it Lecturer: #2 \hfill} }
\vspace{2mm} \hbox to 6.28in { {\it Student: #3 \hfill} }
\vspace{2mm} } }
\end{center}
\markboth{#1}{#1} \vspace*{4mm} }


\begin{document}

\homework{Bayesian Methods}{Changshui Zhang
\hspace{5mm} {\tt zcs@mail.tsinghua.edu.cn}}{ \hspace{5mm} {\tt
 } }{8}

%%%%%%%%%%%%%%%%%%%%%%%%%%%%%%%%%%%%%%%%%%%%%%%%%%%%%%%%%%%%%%%%%%%%
% Section 2.  Problem
%%%%%%%%%%%%%%%%%%%%%%%%%%%%%%%%%%%%%%%%%%%%%%%%%%%%%%%%%%%%%%%%%%%%

\section*{MLE and MAP}

Maximum Likelihood Estimation (MLE) and Maximum A Posterior (MAP) are two basic principles for
learning parametric distributions. In this problem you will derive the MLE and the MAP estimates for some
widely-used distributions.

Before stating the problems, we first give a brief review of MLE and MAP. Suppose we consider a family of
distributions (c.d.f or p.m.f.) $F:=\{f(x|\theta):\theta\in\Theta\}$, where x denotes the random vector, $\theta$
denotes a vector of parameters, and $\Theta$ denotes the set of all possible values of $\theta$. Given a set
$\{x_1,x_2,...,x_n\}$ of sample points independently drawn from some $f^*\in F$, or equivalently some $f(x|\theta^*)$
such that $\theta^*\in \Theta$, we want to obtain an estimate of the value of $\theta^*$. Recall that in the case
of an independently and identically distributed(i.i.d.) sample the log-likelihood function is in the following
form
\begin{equation}
l(\theta)=\sum_{i=1}^n \log f(x_i|\theta),
\end{equation}
which is a function of $\theta$ under some fixed sample $\{x_1,x_2,...,x_n\}$. The MLE estimate $\hat{\theta}_{mle}$ is
then defined as follows:
\begin{itemize}
  \item $\hat{\theta}_{mle}\in\Theta$,
  \item $\forall \theta\in\Theta,$ $l(\theta)\leq l(\hat{\theta}_{mle})$.
\end{itemize}

If we have access to some prior distribution $P(\theta)$ over $\Theta$, be it from past experiences or domain knowledge
or simply belief, we can think about the posterior distribution over $\Theta$:
\begin{equation}
q(\theta):=\frac{\left(\prod_{i=1}^nf(x_t|\theta)\right)p(\theta)}{z(x_1,x_2,...,x_n)},
\end{equation}
where
\begin{equation}
z(x_1,x_2,...,x_n):=\int_\Theta \left(\prod_{i=1}^nf(x_t|\theta)\right)p(\theta) d\theta.
\end{equation}
The MAP estimate $\hat{\theta}_{map}$ is then defined as follows:
\begin{itemize}
  \item $\hat{\theta}_{map}\in \Theta$,
  \item $\forall \theta \in \Theta,$ $q(\theta)\leq q(\hat{\theta}_{map})$, or equivalently,
  \begin{equation}
  l(\theta)+\log p(\theta) \leq l(\hat{\theta}_{map}) + \log p(\hat{\theta}_{map}).
  \end{equation}
\end{itemize}

1. MLE for the uniform distribution

Consider a uniform distribution centered on 0 with width $2a$. The density function is given by

\begin{equation}
p(x) = \frac{1}{2a}I(x \in [-a, a])
\end{equation}


a. Given a data set $x_1, x_2, \cdots , x_n$, what is the maximum likelihood estimate of $a$( call it $\hat{a}$)

b. What probability would the model assign to a new data point $x_{n+1}$ using $\hat{a}$

c. Do you see any problem with the above approach? Briefly suggest a better approach



2. Consider a training data of $N$ i.i.d. (independently and identically distribute) observations, $\bm X=\{x_1, x_2, ..., x_N\}$ with corresponding $N$ target values $\bm T=\{t_1, t_2, ..., t_N\}$.

We want to fit these observations into some model
\begin{equation}\label{eq1}
t = y(x, \bm w) + \epsilon
\end{equation}

where $\bm w$ is the model parameters and $\epsilon$ is some error term.

2.1 To find $\bm w$, we can minimize the sum of square error
\begin{equation} \label{eq2}
E(\bm w) = \frac{1}{2}\sum_{n=1}^N\{y(x_n, \bm w)-t_n\}^2
\end{equation}

Now suppose we believe that the distribution of error term $\epsilon$ is gaussian
\begin{equation}\label{eq3}
p(\epsilon|\beta) = \mathcal N(\epsilon|0, \beta^{-1})
\end{equation}

where $\beta = \frac{1}{\sigma^2}$ is the inverse of variance. Using the property of gaussian distribution, we have
\begin{equation}\label{eq4}
p(t|x, \bm w, \beta) = \mathcal N(t|y(x, \bm w), \beta^{-1})
\end{equation}

Under this assumption, the likelihood function is given by

\begin{equation}\label{eq5}
p(\bm T|\bm X, \bm w, \beta) = \prod_{n=1}^N \mathcal N(t_n|y(x_n, \bm w), \beta^{-1})
\end{equation}

Show that the problem of finding the maximum likelihood (ML) solution for $\bm w$ is equivalent to the problem of minimizing the sum of square error (\ref{eq2}).

2.2 In order to avoid overfitting, we often add a weight decay term to (\ref{eq2})

\begin{equation}\label{eq6}
  E(\bm w) = \frac{1}{2}\sum_{n=1}^N\{y(x_n, \bm w)-t_n\}^2 + \frac{\lambda}{2}||\bm w||^2
\end{equation}

On the other hand, we believe that $\bm w$ has a prior distribution of
\begin{equation}\label{eq7}
  p(\bm w|\alpha) = \mathcal N(\bm w|\bm 0, \alpha^{-1}\bm I)
\end{equation}

Using Bayes’ theorem, the posterior distribution for $\bm w$ is proportional to the product of the prior distribution and the likelihood function
\begin{equation}\label{eq8}
  p(\bm w|\bm X, \bm T, \alpha, \beta) \propto p(\bm T|\bm X, \bm w, \beta)p(\bm w|\alpha)
\end{equation}

Show that the problem of finding the maximum of the posterior (MAP) solution for $\bm w$ is equivalent to the problem of minimizing (\ref{eq6}).


\section*{Naive Bayes}

1. Considers the learning function $X \rightarrow Y$, where class label $Y\in\{T,F\}$, $X=\langle X_1,X_2,...,X_n \rangle$ where $X_1$ is a boolean variable and $\{X_2,...,X_n\}$ are continuous variables. Assume that for each continuous $X_i$, $P(X_i|Y=y)$ follows a Gaussian distribution. List and give the total number of the parameters that you would need to estimate in order to classify a future example using a Naive Bayes classifier.
Give the formula for computing $P(Y|X)$ in terms of these
parameters and feature variables $X_i$.

2. Consider a simple learning problem of determining whether Alice and Bob from CA will go to hiking or not
$Y:Hike\in\{T,F\}$ given the weather conditions $X_1:Sunny\in\{T,F\}$ and $X_2:Windy\in\{T,F\}$ by a
Naive Bayes classifier. Using training data, we estimated the parameters $P(Hike) = 0.5$, $P(Sunny|Hike) = 0.9$, $P(Windy|\neg Hike) = 0.8$, $P(Windy|Hike) = 0.3$ and $P(Sunny|\neg Hike) = 0.4$. Assume that the true distribution of $X_1$, $X_2$, and $Y$ satisfies the Naive Bayes
assumption of conditional independence with the above parameters.

2.1. Assume Sunny and Windy are truly independent given Hike. Write down the Naive Bayes
decision rule for this problem using both attributes Sunny and Windy.

2.2. Given the decision rule above, what is the expected error rate of the Naive Bayes classifier? (The
expected error rate is the probability that each class generates an observation where the decision
rule is incorrect.)

2.3. What is the joint probability that Alice and Bob go to hiking and the weather is sunny and
windy, that is $P(Sunny,Windy,Hike)$?

2.4. Next, suppose that we gather more information about weather conditions and introduce a new feature
denoting whether the weather is $X_3$: Rainy or not. Assume that each day the weather in CA can
be either Rainy or Sunny. That is, it can not be both Sunny and Rainy (similarly, it can not be
$\neg Sunny$ and $\neg Rainy$). In the above new case, are any of the Naive Bayes assumptions violated? Why (not)? What is
the joint probability that Alice and Bob go to hiking and the weather is sunny, windy and not
rainy, that is $P(Sunny,Windy,\neg Rainy,Hike)$?

2.5. What is the expected error rate when the Naive Bayes classifier uses all three attributes? Does
the performance of Naive Bayes improve by observing the new attribute Rainy? Explain why.





\section*{Programming}
In this problem you will implement Naive Bayes and Logistic Regression, then compare their performance on a
document classification task. The data for this task is taken from the 20 Newsgroups data set, and is available
from the attached zip file. The included README.txt describes the data set and file format.

Our Naive Bayes model will use the bag-of-words assumption. This model assumes that each word in a
document is drawn independently from a multinomial distribution over possible words. (A multinomial
distribution is a generalization of a Bernoulli distribution to multiple values.) Although this model ignores
the ordering of words in a document, it works surprisingly well for a number of tasks. We number the words
in our vocabulary from 1 to $m$, where $m$ is the total number of distinct words in all of the documents.
Documents from class $y$ are drawn from a class-specific multinomial distribution parameterized by $\theta_y$. $\theta_y$ is
a vector, where $\theta_{y,i}$ is the probability of drawing word $i$ and $\sum_{i=1}^m \theta_{y,i}=1$.
Therefore, the class-conditional probability of drawing document x from our Naive Bayes model is
$P(X = x|Y = y) = \prod_{i=1}^m (\theta_{y,i})^{count_i(x)}$, where $count_i(x)$ is the number of times word $i$ appears in $x$.

1. Provide high-level descriptions of the Naive Bayes and Logistic Regression algorithms. Be
sure to describe how to estimate the model parameters and how to classify a new example.

2. Imagine that a certain word is never observed in the training data, but occurs in a test
instance. What will happen when our Naive Bayes classifier predicts the probability of the this test
instance? Explain why this situation is undesirable. How to avoid this problem? Will logistic regression have a similar problem?
Why or why not?

3. Implement Logistic Regression and Naive Bayes. Use add-one smoothing when estimating
the parameters of your Naive Bayes classifier. For logistic regression, we found that a step size around
0.0001 worked well. Train both models on the provided training data and predict the labels of the test
data. Report the training and test error of both models. Submit your code along with your homework.

4. Which model performs better on this task? Why do you think this is the case?


%%%%%%%%%%%%%%%%%%%%%%%%%%%%%%%%%%%%%%%%%%%%%%%%%%%%%%%%%%%%%%%%%%%%
% Reference
%%%%%%%%%%%%%%%%%%%%%%%%%%%%%%%%%%%%%%%%%%%%%%%%%%%%%%%%%%%%%%%%%%%%
%\begin{thebibliography}{1}

%\bibitem{BoydVandenberghe2004}
%S. Boyd and L. Vandenberghe, \emph{Convex Optimization}, Cambridge
%University Press, 2004.

%\end{thebibliography}
\end{document}
